% Created 2023-11-27 Mon 11:37
% Intended LaTeX compiler: pdflatex
\documentclass[11pt]{article}
\usepackage[utf8]{inputenc}
\usepackage[T1]{fontenc}
\usepackage{graphicx}
\usepackage{longtable}
\usepackage{wrapfig}
\usepackage{rotating}
\usepackage[normalem]{ulem}
\usepackage{amsmath}
\usepackage{amssymb}
\usepackage{capt-of}
\usepackage{hyperref}
\author{exaclior}
\date{\today}
\title{Final Exam Problem Bank}
\hypersetup{
 pdfauthor={exaclior},
 pdftitle={Final Exam Problem Bank},
 pdfkeywords={},
 pdfsubject={},
 pdfcreator={Emacs 29.1 (Org mode 9.7)}, 
 pdflang={English}}
\begin{document}

\maketitle
\tableofcontents

\section{Purely Linear Algebra}
\label{sec:org4052b47}
\subsection{MAT 310 II}
\label{sec:org7204c28}
\subsubsection{P2: How to take determinant}
\label{sec:org528afea}
(15pts) Let \[ A=\left(\begin{array}{cccc} 4 & 3 & 1 & 2 \\ 1 & 9 & 0 & 2 \\ 8 &
3 & 2 & -2 \\ 4 & 3 & 1 & 1 \end{array}\right) \]
\begin{enumerate}
\item Calculate the determinant of \(A\) using any method that you know.
\item What is the determinant of \(-2 A\) ?
\end{enumerate}
\subsubsection{P3: Determinant and Invertibility}
\label{sec:org28c7bc4}
(13pts) Let \(A\) and \(B\) be \(n \times n\) matrices such that \(A B=-B A\).
Prove that if \(n\) is odd, then \(A\) or \(B\) is not invertible.
\subsubsection{P4: Diagonalizing a matrix}
\label{sec:org2b0a90c}
(13pts) Determine if the following matrix is diagonalizable and justify your
answer. If so, find an invertible matrix \(Q\) and a diagonal matrix \(D\) such
that \(A=Q D Q^{-1}\).
\[ A=\left(\begin{array}{lll} 1 & 1 & 1 \\ 0 & 1 & 0 \\ 0
& 0 & 1 \end{array}\right) \]
\subsection{MAT 310 I}
\label{sec:org721d171}
\subsubsection{P3: Basis}
\label{sec:org71f5799}
(15 pts) Determine whether or not \(\{(1,1,0),(2,0,-1),(-3,1,1)\}\) is a basis
for \(\mathbb{R}^{3}\).
\subsubsection{P4: Rank Nullity Theorem}
\label{sec:orgcd0e7c9}
(15pts) Let \(T: \mathbb{R}^{4} \rightarrow \mathbb{R}^{3}\) denote a linear
transformation such that \(T((1,0,0,0))= (3,-1,0), T((1,1,1,1))=(-2,1,3)\), and
\(T((0,0,1,1))=(0,1,1)\). Compute the dimension of the null space
\(\operatorname{dim}(N(T))\). Hint:\(dim(\mathcc{R}^{4}) = rank(T) + null(T)\)
\subsection{NYU Final}
\label{sec:org3f34889}
\subsubsection{P5: Rank Nullity Theorem}
\label{sec:org6fbb559}
\begin{enumerate}
\item Suppose \(A\) is an \(8 \times 7\) matrix in which \(\operatorname{dim}(\operatorname{Nul} A)=6\). Then rank \(A=\)
\end{enumerate}
(a) 1
(b) 2
(c) 6
(d) 7
(e) 8
\subsubsection{P7: Invertibility}
\label{sec:orgdb3c8bd}
Consider the \(3 \times 3\) matrix \(A=\left[\begin{array}{rrr}0 & 1 & k \\ 2 & k
 & -6 \\ 2 & 7 & 4\end{array}\right]\). For what values of \(k\) is matrix \(A\)
invertible?

(a) \(k \in \mathbb{R}\)
(b) all real \(k\) except 2 and 5
(c) \(k \geq 0\)
(d) \(k=0\)
(e) no value of \(k\) makes \(A\) invertible
\subsubsection{P8: Definition of Eigenvector}
\label{sec:orga2e7a96}
 \(\left[\begin{array}{l}1 \\ 2 \\ 2\end{array}\right]\) is an eigenvector of
   \(\left[\begin{array}{rrr}4 & -2 & 1 \\ 2 & 0 & 1 \\ 2 & -2 &
   3\end{array}\right]\). What is the corresponding eigenvalue?
(a) 0
(b) 2
(c) 3
(d) 7
(e) need more information to determine eigenvalue associated with eigenvector \(\left[\begin{array}{l}1 \\ 2 \\ 2\end{array}\right]\)
\subsubsection{P13: Exponentiating a matrix}
\label{sec:orge6b554d}
(10 points) Compute \(A^{2017}\) where \(A=\left[\begin{array}{rr}1 & 0 \\ 2 &
    -1\end{array}\right]\). (Hint: Diagonalize \(A\).)
\section{General}
\label{sec:org0a5e3ab}
\subsection{Sakurai}
\label{sec:org1b35d94}
\subsection{Likharev}
\label{sec:orge08b5c0}
\begin{itemize}
\item 2.1
\item 4.4
\item 4.9
\item 4.19
\end{itemize}
\section{Quantum Information}
\label{sec:org9faf083}
\subsection{Bacon Final}
\label{sec:org868da5d}
\begin{itemize}
\item P1
\item P2
\item P3
\item P4
\end{itemize}
\subsection{MIT P3}
\label{sec:org604be0b}
\begin{itemize}
\item P1
\item P4
\item P5
\end{itemize}
\subsection{ETH P3}
\label{sec:orgbae5696}
\begin{itemize}
\item Shor's code
\end{itemize}
\section{Atomic Physics}
\label{sec:org289fa71}
\subsection{The Quantum Mechanics Solver}
\label{sec:org0c85a47}
\section{Condensed Matter Physics}
\label{sec:orgf0aef5e}
\begin{itemize}
\item Provided by Prof. Liu
\end{itemize}
\end{document}